\documentclass[spanish]{article}

%M\'argenes piolas
\addtolength{\oddsidemargin}{-.5in}
\addtolength{\evensidemargin}{-.5in}
\addtolength{\textwidth}{1in}
\addtolength{\topmargin}{-.5in}
\addtolength{\textheight}{1in}

\usepackage[T1]{fontenc}

\usepackage{babel}
\usepackage[utf8]{inputenc}
\usepackage{amsfonts}
\usepackage{amsmath,amssymb}
\usepackage[shortlabels]{enumitem}
\usepackage{mathrsfs}
\usepackage{hyperref}
\hypersetup{
      colorlinks,
      citecolor=black,
      filecolor=black,
      linkcolor=black,
      urlcolor=black
}
\usepackage{xcolor}

%no section numbering
\setcounter{secnumdepth}{0}

\newcommand{\RR}{\mathbb{R}}


%% README:
%% Traduccion de lo programas de las materias de Matematica y Computacion de la UBA
%% Es probable que esto contenga errores. Tambien falta calculo numerico para matematicos.

\begin{document}

\section{General remarks}
\begin{itemize}
  \item
  The recommended bibliography of the courses usually contains four or five books
  but I include only the ones that I consulted during the course.

  \item
  ``Chapters $x$ to $y$'' means that chapter $y$ is included.

  \item
  I had to manually translate the outlines, so in this sense they are not
  official.
\end{itemize}

\section{Department of Mathematics -- required courses}

\subsection{Algebra I}

\begin{itemize}
  \item
Operations with sets. Properties, De Morgan law. Cartesian product. Functions,
graphic, bijections. Composition. Relations. Equivalence relations. Partitions.
Quotient sets.

  \item
Induction. Inductive definitions.

  \item
Elements of combinatorial analysis. Combinations, permutations. Combinations
with repetition. Partitions.

  \item
Integers, divisibility. Division algorithm. Greatest common divisor and Least
common multiple. Prime numbers. Fundamental theorem of arithmetic. Factorization.
Congruences. Numeration systems. Rational and irrational numbers.

  \item
Complex numbers. Trigonometric form. De Moivre theorem. Roots of unity.

  \item
Polynomials. Polynomial remainder theorem. Divisibility. Roots and multiplicity.
Gauss's lemma (polynomials).
\end{itemize}

\subsubsection{Bibliography}
\begin{itemize}
  \item
E. Gentile. Notas de \'Algebra. Eudeba, Buenos Aires.\\
Complete.

  \item
Birkhoff-Mc Lane. A Survey of Modern Algebra.\\
Chapter $1$. Chapters $2$ to $5$ optional, to get a deeper understanding of
the contents of the course.
\end{itemize}

\hrulefill
%\hrulefill%------------------------------

\subsection{Linear Algebra}

\begin{itemize}
  \item
The vector spaces $K^n$. Linear dependence. Systems of linear equations. Matrix
notation. Gauss's method. Linear dependence of rows and columns. Basic results:
homogeneous systems with more unknowns than equations.

  \item
Permutations. Symmetric Group. Determinant and fundamental properties. Computation
of determinant, minors, cofactors. Laplace's theorem, computation of determinant
by rows and by columns.

  \item
The matrix ring. Determinant of a product. Adjoint matrix. Inverse matrix. Cramer's rule.
Row and column operations. Elemental matrices. General Linear Group. Equivalence
of matrices. Rank.


  \item
Abstract vector spaces. Generators. Basis. Existence of basis in the finite dimensional case.
Coordinates and isomorphism with $K^n$. Subspaces and linear varieties. Intersection, sum.
Theorem of direct sum (dimension). Linear forms. Dual space. Dual basis. Annihilator.

  \item
Linear transformations. Sum and composition. Associated matrix (choosing a basis).
Matrix of a composition. Isomorphism with the matrix ring. Change of basis matrix
and change of coordinates. Similarity of matrices. Kernel, Image and Dimension theorem
(kernel extension theorem for vector spaces). Eigenvalues and eigenvectors. Characteristic
polynomial. Diagonalization.


  \item
Bilinear forms. Associated matrix (choosing a basis). Symmetric bilinear forms.
Quadratic forms. Polarization identity. Real case. Positive-definite forms characterization.
Complex case. Hermitian forms.


  \item
Scalar product, angle, cosine, projection of vectors, orthogonality. Distance between linear
varieties. Scalar product in abstract spaces. Real and complex case. Cauchy-Schwarz.
Gram-Schmidt. Orthonormal basis.

  \item
Adjoint transformation. Rotations in $R^n$. Structure of orthogonal transformations.
Diagonalization with orthogonal matrices.


  \item
Nilpotent operators. Base and Jordan canonical form. Minimal polynomial. Invariant subspaces.
Triangular form. Primary decomposition. Cayley-Hamilton. Basis and Jordan form.

\end{itemize}



\subsubsection{Bibliography}
\begin{itemize}
  \item
Seymour Lipschutz, Marc Lipson. Schaum's Outline of Linear Algebra.\\
Complete.

  \item
Gabriela Jeronimo, Juan Sabia, Susana Tesauri. \'Algebra Lineal.
Cursos de Grado - Publicaciones del departamento de Matem\'atica.\\
Complete.
\end{itemize}


\hrulefill%------------------------------


\subsection{Algebra II}


\begin{itemize}
  \item
Groups.
Binary operations. Monoids. Semigroups. Groups. Morphisms. Quotient by subgroup.
Compatible equivalence relations and normal subgroups. Examples: cyclic groups,
symmetric groups, alternate groups, matrix groups, automorphism group of some
structure. Semidirect product. Actions of groups, orbits, quotient by action.
Sylow's theorems.


  \item
Rings.
Morphisms. Ideals. Quotient ring. Examples. Zero divisors. Nilpotent elements.
Unities. Irreducible elements. Prime ideals, maximal ideals. Euclidean domains.
Principal ideal domain. Unique factorization domains.


  \item
Modules.
Examples: vector spaces, abelian groups, ideals, endomorphism space of vector space,
linear representations. Morphisms. Submodules and quotient module. Operations
with submodules. Isomorphism theorems. Exact sequences, commutative diagrams.
Projective and injective modules.
Direct sum and product. Finitely generated modules. Free modules. Torsion. Divisibility.
Structure of torsion and divisible modules over a principal ideal domain. Multiplicative
sets. Fraction ring and module, localization. Noetherian and artinian modules.
Hilbert's basis theorem. Structure of finitely generated modules over a principal
ideal domain. Tensor product. Scalar extension and restriction. Multilinear algebra.
Graded algebras. Tensor, symmetric and exterior algebra of a module.
Semisimple rings and modules. Examples.

\end{itemize}


\subsubsection{Bibliography}
\begin{itemize}
  \item
Atiyah-Macdonald. Introduction to Commutative Algebra\\
Chapters $1$ to $4$.

  \item
Lang, Serge: Algebra, Addison-Wesley, Reading, 1965.\\
Chapters $1$ to $4$. Chapters $16$ and $17$.
\end{itemize}


\hrulefill%------------------------------

\subsection{Algebra III}


\begin{itemize}
  \item
Fields and extensions. Fraction fields. Characteristic, prime fields.

  \item
Polynomials and rational fractions. Universal algebra of a semigroup.
Polynomial algebra, algebraic dependence.

  \item
Polynomial factorization. Primitive polynomials, Gauss's lemma. Eisenstein's criterion.

  \item
Finite extensions and simple extensions. Distinguished class of extensions.

  \item
Algebraic extensions. Algebraic elements, minimal polynomial. Transcendental elements.
Transcendental extensions.

  \item
Algebraically closed fields. Algebraic closure.
Existence of algebraic closure.

  \item
Decomposition field. Existence and uniqueness.

  \item
Group representations, conjugation and orbits. Endomorphisms of algebraic
extensions. Conjugated fields.

  \item
Normal extensions. Scalar extension.

  \item
Separable extensions. Separable elements. Finite separable extensions. Separable
polynomials. Primitive element theorem.

  \item
Galois extensions. Scalar extension.

  \item
Galois theory. Normal subextensions. Finite groups of automorphisms. Artin's theorem.
Galois's fundamental theorem.

  \item
Radical extensions. Radical elements. Radical closure.

  \item
Purely inseparable extensions. Separable closure. Separability and inseparability
degree.

  \item
Perfect fields.

  \item
Trace and norm. Separability and trace. Discriminant of the trace form in separable
extensions.

  \item
Introduction to Galois cohomology. Algebraic independence. Normal basis theorem for
Galois extensions. Hilbert's theorem $90$.

  \item
Abelian extensions and cyclic extensions. Normal basis for cyclic extensions.
Quadratic extensions.

  \item
Finite fields. Classification. Finite extensions over finite fields. Generators
of the Galois group. Surjectivity of norm and trace.

  \item
Roots of unity. Structure and properties of the roots of unity groups in a field.
Primitive roots.

  \item
Cyclotomic fields. Structure of the unity group of integers modulo $n$. Cyclotomic
polynomials. Irreducibility criteria.

  \item
Cyclic extensions. Abelian extensions of degree $p$, in characteristic $p$ and
Artin-Schreier equations.

\end{itemize}


\subsubsection{Bibliography}
\begin{itemize}
  \item
Lang, Serge: Algebra, Addison-Wesley, Reading, 1965.\\
Chapters $5$, $6$ and $8$.

  \item
Gentile, E.R.: Teoría de cuerpos, Notas de Matemática, IMAF (Universidad de Córdoba). Córdoba, 1969.\\
Complete.
\end{itemize}


\hrulefill%------------------------------

\subsection{Analysis I}

\begin{itemize}
  \item
Completeness of $\RR$. Supreme's existence and equivalences. Distance, open disks and closed disks. Interior points. Interior of a set. Open sets. Adherent points. Closure of a set. Closed sets. Bounded sets. Limit of real numbers' sequences. Limit of sequences in $\RR^n$ and limit in each coordinate.

  \item
Functions from $\RR^n$ to $\RR^k$. Graphical representation. Domain of definition. Level curves and surfaces. Limit of functions from $\RR^n$ to $\RR^k$. Limit along lines and curves. Continuous functions. Composition of continuous functions. Properties of continuous functions.

  \item
Partial derivatives. Linear approximation. Differential of a function. Jacobian matrix. Tangent plane to the graph of a function. Chain rule. General theorems of the inverse function and of the implicit function. Scalar product in $\RR^n$. Equation of a plane orthogonal to a vector. Directional derivatives. Gradient. Relation with the level surfaces and the direction of maximal growth. Tangent plane to a level surface. Mean value theorem in several variables. Higher order derivatives. Polynomial approximation of second order. Hessian matrix (or Hessian) of a function.

  \item
Critical points and extrema of a function. Quadratic forms, associated matrix. Analysis of the critical points in several variables through the Hessian: maxima, minima, saddle points. Constrained extrema: extrema of a function over a set given by an equation $G = 0$. Condition for a point to be a critical point. Lagrange multipliers.

  \item
Review: definite integral, Riemann sums, Fundamental theorem of calculus, Barrow's rule. Improper integrals: definitions, properties, convergence criteria, absolute convergence. Application: convergence of series. The double integral over rectangles. The double integral over more general domains. Change of the order of integration: Fubini's theorem. The triple integral. The change of variables theorem. Applications of double and triple integrals.

\end{itemize}

\subsubsection{Bibliography}
\begin{itemize}
  \item J. Marsden and A. Tromba. Vector Calculus. Freeman and Company.\\
    Complete.

  \item T. Apostol. Calculus, Vol. I. John Wiley \& Sons.\\
    Chapters $1$ to $5$.

\end{itemize}



\hrulefill%------------------------------
\subsection{Analysis II}

\begin{itemize}
  \item
The line integral. Parametrized surfaces. Area of a surface. Integrals of scalar functions over surfaces. Integrals of vector fields over surfaces. Applications.

  \item
Green's theorem. Stokes' theorem. Conservative fields. Gauss' theorem. Applications.

  \item
Differential equations. Introduction and elementary methods. Existence and uniqueness theorem. Maximal solutions. Systems of first order linear differential equations and higher order differential equations.

  \item
Resolution of systems of linear differential equations with constant coefficients. Field lines. Linear stability. Conservative systems. Applications.
\end{itemize}

\subsubsection{Bibliography}
\begin{itemize}
  \item Noem\'i Wolansky. Introducci\'on a las Ecuaciones Diferenciales Ordinarias.
    Cursos de Grado - Publicaciones del departamento de Matem\'atica.\\
    Complete.

  \item J. Rey Pastor, P. Pi Calleja and C. Trejo. Análisis Matemático, Vol. II. Ed. Kapelusz.\\
    Chapters $20$ and $23$.
\end{itemize}


\hrulefill%------------------------------

\subsection{Advanced Calculus}

\begin{itemize}
  \item
Real numbers. Sequences. Monotone, bounded and Cauchy sequences.
Extended real line. Superior and inferior limit. Series of positive
terms. b-ary notation, uniqueness and non uniqueness.

  \item
Cardinality. Equivalence of sets. Finite and infinite sets. Countable
sets. Uncountable sets. Continuum. Schr\"oder-Bernstein. Cantor's theorem.
Operations between cardinals.

  \item
Metric spaces. Distances. Open and closed balls. Interior. Accumulation points.
Neighborhood of a point. Open and closed sets. Limits and continuous functions.
Diameter and distance of sets. Subspaces. Bounded and totally bounded sets.
Dense sets and separable spaces. Completeness. Compactness. Baire category
theorem. Homeomorphisms. Equivalent metrics. Isometries. Connected spaces and sets.
Banach fixed-point theorem.

  \item
Normed spaces. Banach spaces. Lineal and continuous maps. Homeomorphisms and
equivalent norms. Sequences and series of functions. Pointwise and uniform
convergence. Uniform convergence and continuity. Uniform convergence and
integration. Uniform convergence and differentiation. Equicontinuous functions.
Arzel\'a-Ascoli theorem. Stone-Weierstrass theorem. Immersion of a space $E$ in
$C(E)$. Cantor-Hausdorff completion theorem.

  \item
Differentiation in euclidean spaces. Differentiable maps. Properties of
the differential. Partial derivatives. Jacobian matrix. Chain rule. Inverse function
theorem. Implicit functions.
\end{itemize}



\subsubsection{Bibliography}
\begin{itemize}
  \item
Kolmogorov, Fomin. Elements of the Theory of Functions and Functional Analysis, Volume $1$,
Metric and Normed Spaces.\\
Chapters $1$ and $2$.

  \item
Dieudonne. Foundations of Modern Analysis.\\
Chapters $1$, $2$, $3$ and $5$.

  \item
Irving Kaplansky. Set Theory and Metric Spaces.\\
Complete.

\end{itemize}

\hrulefill%------------------------------
\subsection{Probability and Statistics [Mathematics]}

\begin{itemize}
  \item
Sample space. Events. Algebra of events. Probability space. Properties. Superior and inferior limit of sets.

  \item
Conditional probability and independence of events. Borel-Cantelli lemma.

  \item
Random variables. Distribution function. Usual distributions. Joint probability distribution. Independence of random variables. Change of variables.

  \item
Expected value of random variables. Properties of the expected value, variance and covariance. Monotone and dominated convergence theorems.

  \item
Conditional probability distribution and expectation. Definition, particular cases and properties.

  \item
Convergence in probability and almost sure convergence. Markov and Chebyshev's inequalities. Weak law of large numbers. Applications. Kolmogorov's inequality. Strong law of large numbers.

  \item
Weak convergence. Definition. Helly's selection theorem. Characteristic functions. Properties. Inversion theorem. Lévy's continuity theorem. Central limit theorem. Applications.
\end{itemize}

\subsubsection{Bibliography}
\begin{itemize}
  \item Sheldon Ross. A First Course In Probability.\\
    Chapters $1$ to $8$ and $9.1$, $9.2$.

  \item Durret. Probability, Theory and Examples.\\
    Chapters $1$ to $3$ and $6$.
\end{itemize}

\hrulefill%------------------------------

\subsection{Complex Analysis}

\begin{itemize}
  \item
Complex numbers. Conjugation. Absolute value. Polar form. Powers and roots. Topology
and continuity. Riemann sphere. Homographies.

  \item
Complex variable functions. Differentiability. Chain rule, derivative of the inverse.
Cauchy-Riemann equations. Armonic functions. Armonic conjugate functions. Conformal maps.

  \item
Sequences and series of complex numbers. Convergence criteria for series.
Function series. Pointwise, absolute, uniform and normal convergence. Weierstrass criterion.
Power series. Abel's lemma. Convergence radius. Analytic functions.

  \item
Elemental functions. The exponential map. Properties and characterization. Trigonometric
functions. Complex logarithm.

  \item
Integration of complex functions. Cauchy-Goursat theorem for rectangles. Cauchy's theorem
for the disc. Morera's theorem. Winding number. Cauchy integral formula. Higher order derivatives.
Cauchy inequalities. Liouville's theorem. Fundamental theorem of algebra.

  \item
Taylor expansion. Holomorphic functions are analytic. Zeroes of analytic functions.
Order of zeroes.

  \item
Maximum modulus principle. Open mapping theorem. Inverse functions. Schwarz lemma.

  \item
General statement of Cauchy's theorem. Simply connected sets. Homotopic curves.

  \item
Isolated singularities. Laurent series. Region of convergence. Classification of
isolated singularities. Study of poles. Casorati-Weierstrass theorem. Singularity
at infinity.

  \item
Residues. Residue theorem. Logarithmic derivative. Rouch\'e's theorem.
Meromorphic functions over the Riemann sphere.

  \item
Uniform convergence over compact sets. The space of holomorphic functions on an
open connected set. Montel's theorem. Series of meromorphic functions.

  \item
Infinite products. Weierstrass theorem.

  \item
Conformal representation. Riemann's fundamental theorem. Biholomorphisms of the
plane, the disc and the semiplane.
\end{itemize}



\subsubsection{Bibliography}
\begin{itemize}
  \item
Ahlfors, L. V: Complex Analysis, Mc.Graw-Hill Book Co. (1979)\\
Chapters $1$ to $5$.

  \item
Conway, J.B: Functions of One Complex Variable, Second edition, Springer-Verlag (1978)
Chapters $1$ to $7$.

\end{itemize}


\hrulefill%------------------------------
\subsection{Real Analysis}


\begin{itemize}
  \item
Lebesgue measure in $\RR^n$. Measure on intervals and sigma-elementary sets. Outer measure.
Measurable sets. Lebesgue measure. Monotonic sequences of measurable sets. Sets of null measure. G-delta and F-sigma sets. Structure of measurable sets. Algebras and sigma-algebras. Borel sets. Translational invariance. Non-measurable sets.

  \item
Measurable functions. Algebraic operations and sequences of measurable functions. Simple functions. Borel functions. Properties true almost everywhere. Egoroff's theorem. Lusin's theorem. Convergence in measure.

  \item
Lebesgue integral. Integral of non-negative functions. Integral of simple functions. Monotonic convergence theorem. Fatou's lemma. Integral of real-valued functions. Linearity. Uniform convergence theorem. Dominated convergence theorem. Chebyshev's inequality. Integral of complex valued functions. Translational invariance. The integral as a set valued function. Absolute continuity of the integral. Comparison against Riemann integration.

  \item
Fubini's theorem. Cavalieri's principle. The Tonelli and Fubini theorems. Convolutions. Distribution functions.

  \item
Change of variables. Image of a measurable set under a linear transformation. Differentiable mappings. Change of variables formula.

  \item
$L^p$ spaces. Hölder and Minkowski's inequalities. Completeness. Dense classes of functions. Separability. Continuity modulus. Convolution. Young's theorem.

  \item
Differentiation of the integral. Simple Vitali's lemma. The Hardy-Littlewood maximal function. Maximal theorem. Lebesgue's differentiation theorem. Vitali's covering theorem. Differentiability of monotonic functions. Functions of bounded variation. Absolutely continuous and singular functions.

  \item
Measure and integration on abstract spaces. Measurable spaces. Measures. Measurable functions. Integration on an abstract measure space.

  \item
Signed measures. Hahn decomposition theorem. Jordan-Hahn decomposition of a measure. Complex-valued measures. Total variation. Absolutely continuous and singular measures. Lebesgue-Radon-Nikodym theorem. Bounded linear functions on $L^p$.

\end{itemize}


\subsubsection{Bibliography}
\begin{itemize}
  \item Folland. Real Analysis - Modern Techniques and their Applications.\\
    Chapters $1$ to $3$ and $6$.

  \item Royden. Real Analysis.\\
    Part One and chapters $11$ and $12$ from Part Three.
\end{itemize}


\hrulefill%------------------------------

\subsection{Functional Analysis}


\begin{itemize}
  \item
Normed spaces, elementary properties and examples. Banach spaces, linear functionals, the Hahn-Banach theorem. Linear operators. The open mapping and closed graph theorems. Uniform boundedness principle. Stone-Weierstrass' theorem. Riesz's representation theorem. $L^p$ spaces. Fourier series: uniform and pointwise convergence. Series of averages, $L^1$ convergence. Fej\'er kernel. Sufficient conditions for pointwise and uniform convergence. Example of a continuous function with divergent Fourier series. Poisson kernel.

  \item
Hilbert spaces, properties and examples. The Riesz lemma. The $H^2$ space. Shift operators, invariant subspaces. Orthonormal systems and bases. Operators in Hilbert spaces, examples. Normal, self-adjoint and positive operators. Projectors.

  \item
Weak topologies. Weak and weak* topologies on a Banach space. Alaoglu's theorem. Reflexivity. Goldstine's lemma. Geometric form of the Hahn-Banach theorem.

  \item
Compact operators. Spectrum of an operator. Spectral properties. Riesz-Fredholm theory. Fredholm's alternative. Application: the Dirichlet problem on a bounded domain in $\RR^3$ with smooth boundary.

  \item
Self-adjoint operators. Spectral properties. Spectral decomposition of a compact, self-adjoint operator. Application: regular Sturm-Liouville systems.

  \item
Functional calculus. Spectral measures. Resolutions of the identity. Spectral theorem for self-adjoint operators. Fourier-Plancherel transform.

\end{itemize}

\subsubsection{Bibliography}
\begin{itemize}
  \item Conway. A Course in Functional Analysis.\\
    Chapters $1$ to $3$ and $5$.
  \item Brezis. Functional Analysis, Soboloev Spaces and Partial Differential Equations.\\
    Chapters $1$ to $6$.
\end{itemize}


\hrulefill%------------------------------
\subsection{Differential Equations}

\begin{itemize}
  \item
Review of Cauchy's theorem for ordinary differential equations. Dependence on the initial condition. Examples of partial differential equations. Problem of the local existence of solutions.

  \item
Calculus of variations in one dimension. First variation and Euler-Lagrange equation. Extremals. Hamiltonian systems. Free boundary and isoperimetric problems. Multiple integrals.

  \item
Methods of separation of variables. Completeness of the system of eigenfunctions. Application to the resolution of boundary value problems for the Laplacian, the heat equation and the wave equation on different domains.

  \item
Harmonic functions. Solution of the Dirichlet problem in $\RR^n$. Green's function and Poisson kernel on the half-space and the sphere. Mean value property. Reciprocal of the mean value property. Maximum principle. Harnack's inequality. Analyticity of the harmonic functions.

  \item
Dirac delta function. Convolution product. Fourier transform. Transform of the convolution. Fourier inversion theorem. Fourier transform in $L2$. Application to the calculus of fundamental solutions and to the resolution of initial value problems for the Laplacian, the wave equation, the heat equation, and the Schrödinger equation.

  \item
The heat operator. The Gauss kernel and its applications. The heat equation in bounded domains. Maximum principle. Regularity. The wave equation in $1$, $2$ and $3$ dimensions.

  \item
    Sobolev spaces $W^{k,p}$. Variational formulation of boundary value problems. Existence and uniqueness of the minimizer in $H1$ for Dirichlet's integral. Regularity of the minimizer. Resolution of uniformly elliptic problems of second order. Compactness of the inclusion of $H1$ and $L2$. Eigenvalues. Application to the resolution of the heat equation on bounded domains.

\end{itemize}

\subsubsection{Bibliography}
\begin{itemize}
  \item Simmons. Differential Equations with Applications and Historical Notes.\\
    Chapters $1$ to $3$. Chapters $6$, $7$ and $12$.

  \item L. Evans. Partial Differential Equations. AMS.\\
    Chapters $5$, $6$ and $8$.

  \item R. Courant and D. Hilbert. Methods of Mathematical Physics, Vol. I. Wiley Interscience.\\
    I used it as a reference book.

\end{itemize}


\hrulefill%------------------------------

\subsection{Projective Geometry}

\begin{itemize}
  \item
Affine space. Affine independence, coordinate systems, linear varieties.
Affine transformations. Bilinear forms. Inner product, orthogonality,
isometries. Volume.

  \item
Projective spaces. Homogeneous coordinates. Conics and quadrics. Classification.

  \item
Curves. Parametrized curves. Regular curves. Tangent vector. Arc length. Curvature
and torsion.

  \item
Surfaces. Parametrizations, charts and atlaces. Regular surfaces. Tangent plane.
Differential functions over surfaces. Vector fields. Differential forms. Orientation.
Gauss map. Isometries. Parallel transport. Geodesics.

  \item
Classification of curves and compact surfaces. Existence of triangulations.
Baricentric subdivision. Genus. Classification of non oriented surfaces.

\end{itemize}



\subsubsection{Bibliography}
\begin{itemize}
  \item
Do Carmo, M.; Differential geometry of curves and surfaces, Prentice Hall\\
Chapters $1$ to $4$.

  \item
Larotonda, A.; Algebra Lineal y Gometría. Eudeba, Buenos Aires.\\
Complete.
\end{itemize}




\hrulefill%------------------------------

\subsection{Topology}

\begin{itemize}
  \item
Ordered sets and well ordered sets. Transfinite induction. Zermelo's theorem (well
ordering of sets of cardinals).

  \item
Topological spaces. Open and closed sets, closure and interior. Neighborhoods.
Basis and subbasis of a topology. Order topology. Metric topology. Nets.
Continuous functions.

  \item
Final and initial topology. Product and box topology. Union of spaces. Subspace
topology. Quotient topology. Fibered products.

  \item
Connection and path-connection. Proper functions. Compact and locally compact
spaces. Alexandroff compactification. Topological groups.

  \item
Separability axioms. Urysohn's lemma.

  \item
Tychonoff theorem. Stone-Cech compactification.

  \item
Function spaces. Exponential topology and exponential law.
Compact open topology. K-spaces.


  \item
Homotopy of functions. Relative homotopy. Homotopy equivalences and homotopy
types. Contractible spaces. Deformation retracts. Cylinders and cones of functions.
Extension of functions to the cone.

  \item
Homotopy between paths and loops. Fundamental grupoid and group. Fibers.
Fibrations. Coverings. Fundamental group of spheres.

  \item
Van Kampen theorem and applications.

  \item
Existence and classification of coverings. Regular coverings and deck transformations
group.

  \item
Introduction to singular and simplicial homology. Singular complex. Simplicial
complexes. Domain invariance theorem. Invariance of dimension. Jordan curve theorem.
Mayer-Vietoris and excision theorem.
\end{itemize}



\subsubsection{Bibliography}
\begin{itemize}
  \item
J. Munkres. Topology, a first course. Prentice-Hall.\\
Chapters $1$ to $5$.

  \item
A. Hatcher. Algebraic Topology. Cambridge University Press.\\
Chapters $1$ and $2$.

  \item
E. Spanier. Algebraic Topology. Mc Graw-Hill.\\
Chapters $1$, $2$ and $4$.

\end{itemize}


\hrulefill%------------------------------


\subsection{Differential Geometry}

\begin{itemize}
  \item
Implicit function theorem. Topological manifolds. Differetiable charts and
atlases. Differential structures. Differential manifolds. Submanifolds
of $\RR^n$.

  \item
Differentiable functions. Curves on manifolds. Tangent vectors and tangent
space.

  \item
Differential of a differentiable function. Immersions and submersions.
Properties and examples. Immersed and submersed submanifolds. Adapted charts.
Regular and critical values. Lie Groups.

  \item
Tangent bundle. Vector fields. Integral curves, existence and uniqueness.
Local flow. Completeness. Uniparametric group of diffeomorphisms.

  \item
Derivations and Lie bracket. Lie derivative. Frobenius' theorem. Cotangent
bundle. $1$-forms.

  \item
Tensors and $k$-forms. Local representations. Tensors. Exterior derivative.

  \item
Partitions of unity. Orientable manifolds. Integration over oriented manifolds.
Manifolds with boundary. Stokes theorem.

  \item
Connections. Covariant derivations. Curvature and torsion tensors. Parallel
transport. Geodesics of a connection. Rimannian manifolds. Volume element.
Levi-Civita connection.

  \item
DeRham theory. DeRham complex and DeRham cohomology.

\end{itemize}


\subsubsection{Bibliography}
\begin{itemize}
  \item
Warner. Foundations of Differentiable Manifolds and Lie Groups. Springer.\\
Chapters $1$ to $4$.

  \item
Tu L.W.; And introduction to Manifolds. Springer.\\
Complete.

\end{itemize}


\hrulefill%------------------------------

\section{Department of Mathematics -- elective courses}

\subsection{Logic and Computability}

\begin{itemize}
  \item
Logic. Formal systems. Propositional calculus and first order predicate calculus.
Syntax and semantics. Valuations and truth tables. Semantic consequence and satisfiability.
Refutation trees. Completeness and compactness theorems.

  \item
Computability. Algorithms and computable functions. G\"odel's language for computable
functions. The halting problem. Primitive recursive functions and recursive functions.
Universal programs. Church thesis. Recursion theorem. Recursive sets and recursively
enumerable sets. Rice's theorem.

\end{itemize}


\subsubsection{Bibliography}
\begin{itemize}
  \item
Davis, Weyunker. Computability, Complexity and Languages. Academic Press.\\
Chapters $1$ to $5$. Chapters $11$ and $12$.

\end{itemize}


\hrulefill%------------------------------

\subsection{Grothendieck-Galois theory}

\begin{itemize}
  \item
In the SGA1 Grothendieck develops a theory for categories together
with a fiber functor with values in finite sets. He considers the
category of continuous, transitive actions of the profinite group
of automorphisms of the fiber. The classical theory
of Artin-Galois and the theory of covering spaces are particular
instances of this theory.

  \item
Deligne and Joyal develop the theory of Tannaka in a categorical way.
It can be seen as a theory for abelian categories together with a functor
with values in the tensorial category of finite dimensional vector spaces.
In this case one considers the automorphisms of the fiber functor. These
form a Hopf algebra, and one can study its category of comodules.

  \item
Some generalizations of Galois theory to topos theory. Locally constant
sheaves and locally connected topoi.

\end{itemize}


\subsubsection{Bibliography}
\begin{itemize}
  \item
E. Dubuc. Localic Galois Theory. (article)
  \item
E. Dubuc. On the Representations Theory of Galois and Atomic Topoi. (article)
  \item
E. Dubuc and Constanza S. de la Vega. On the Galois Theory of Grothendieck. (article)

\end{itemize}


\hrulefill%------------------------------

\subsection{Algebraic Topology}

\begin{itemize}
  \item
Adjunction spaces. Cells, cellular spaces, CW complexes. Simplicial complexes.
Fundamental group after attaching cells. Fundamental group of a CW complex.

  \item
Singular homology and cellular homology. Classical results. Lefschetz number and
fixed-point theorems.

  \item
Homotopy groups of higher order. Relations between homotopy and homology. Weak
equivalences. Whitehead's theorem. CW-approximation. Homotopic excision.
Hurewicz's theorm.

  \item
Homology with coefficients. Cohomology and universal coefficients theorem.

  \item
Cellular complexes of dimension $2$ in terms of group presentations.
Topological deformations and transformations in presentations. Open problems
on $2$ dimensional polyhedra and the relation with combinatorial group theory.

\end{itemize}


\subsubsection{Bibliography}
\begin{itemize}
  \item
A. Hatcher. Algebraic Topology. Cambridge University Press.\\
Chapters $2$ to $4$.

  \item
E. Spanier. Algebraic Topology. Mc Graw-Hill.\\
Chapters $3$ to $7$.

\end{itemize}


\hrulefill%------------------------------

\subsection{Basic Topics in Category Theory}

\begin{itemize}
  \item
Categories. Final and initial objects. Duality

  \item
Products, coproducts, equalizers and coequalizers.

  \item
Fibered products, monomorphisms, epimorphisms. Effective epimorphisms.

  \item
Filtrant colimits.

  \item
Limits and colimits.

  \item
Functors and natural transformations. Representable functors.
Adjoint functors. Criteria for the existence of adjoints.

  \item
Presheaf categories. Yoneda's lemma. Generators.
Exactness properties of presheaf categories.

\end{itemize}


\subsubsection{Bibliography}
\begin{itemize}
  \item
Saunders Mac Lane. Categories for the Working Mathematician.\\
Chapters $1$ to $4$.

  \item
Course notes.

\end{itemize}


\hrulefill%------------------------------

\subsection{Differential Topology}

\begin{itemize}
  \item
Critical and regular values. Sard's theorem. Transversality.

  \item
DeRahm cohomology. Sheaves and presheaves. Sheaf cohomology and classical
theories. DeRham's theorem.

  \item
Morse theory. Critical points and Hessian. Morse functions. Fundamental
theorems of Morse theory. Associated cell structure.

  \item
Applications of Morse theory. Characterization of spheres and discs.
Poincar\'e-Hopf index theorem. Classification of compact surfaces.

  \item
Cobordism and h-cobordism. Poincar\'e's conjecture.

  \item
Knots and links. The group of a link. Seifert surfaces. Linking number.

\end{itemize}



\subsubsection{Bibliography}
\begin{itemize}
  \item
Milnor. Topology from the Differentiable Viewpoint.\\
Chapters $1$ to $6$.

  \item
Guillemin, Pollack. Differential Topology.\\
Chapters $1$ to $3$.

  \item
Rolfsen. Knots and Links.\\
Chapters $2$ and $5$.

\end{itemize}

\end{document}
